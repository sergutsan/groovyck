\documentclass{article}
\usepackage[margin=2cm]{geometry}
\usepackage[dvips]{graphicx}
\begin{document}

\section*{Learning goals}
\label{sec:learning-goals}

Before the next day, you should have achieved the following learning
goals: 

\begin{itemize}
\item Read from text files
\item Write to text files
\end{itemize}

\section{ls}
\label{sec:ls}

Write a program that shows on screen the names of the files in the
current directory. (Hint: look at methods from class File). 

\section{mkdir}
\label{sec:mkdir}

Write a program that takes a name from the user at the command line
and creates a directory with that name. Remember that the only
argument of method \verb+main+ is an array of Strings, each of them
an argument written after the name of the class. For example, if you
write \verb+java myClass this That 0+, the elements in \verb+args+
will be three strings of values ``this'', ``That'', and ``0''. 

\section{cat}
\label{sec:cat}

\subsection*{a) }

Write a program that takes a name from the user at the command
line. If a file with that name exists, the program must show its
contents on screen. If it does not, the program must say so. 

\subsection*{b) (*)}
\label{sec:b}

Modify the program so that it takes many file names at the command
line, and then shows them all one after the other. 

\section{cp}
\label{sec:cp}

\subsection*{a) }
\label{sec}

Write a program that takes two names from the user at the command
line. If a file with the first name exists, the program must copy it
into a file with the second name. 

If the first file does not exist, the program must say so. If the
second file does exists, the program must ask the user whether to
overwrite it or not, and act accordingly.

% A solution
%
%   public static void main(String[] args) throws Exception {
%        File src = new File(args[0]);
%        File dst = new File(args[1]);
%        if (dst.exists()) {
%             System.out.println("Overwrite " + dst + " [y/N]?");
%             String yn = System.console().readLine();
%             if (!yn.equals("y")) {
%                  System.out.println("Aborting copy...");
%                  System.exit(0);
%             }
%        }
%        BufferedReader in = new BufferedReader(new FileReader(src));
%        PrintWriter out = new PrintWriter(dst);
%        String line;
%        while ((line = in.readLine()) != null) {
%             out.println(line);
%        }
%        in.close();
%        out.close();
%        System.out.println(src + " -> " + dst);
%   }

\subsection*{b) (*)}
\label{sec:brr}

Modify the program so that it takes many file names at the command
line. When this happens, the last name must be a directory (otherwise,
your program should complain). If it is a directory, your program has
to copy all files (i.e.~the other arguments) into that directory. 

\section{tr (*)}
\label{sec:tr}

Write a program that takes a name and two strings from the user at the
command line. If a file with that name exists, the program must show
its contents on screen, but substituting any occurrence of the first
string by the second string. If the file does not exist, the program
must say so.

\section{sort (*)}
\label{sec:sort-}

Write a program that takes a name from the user at the command
line. If a file with that name exists, the program must show its
contents on screen, but with the lines shown alphabetically. If the
does not exist, the program must say so. 

Hint: Remember that Strings in Java implement the interface
\verb+Comparable<String>+. 

\section{uniq (*)}
\label{sec:uniq-}

Write a program that takes a name from the user at the command
line. If a file with that name exists, the program must show its
contents on screen, but removing duplicates lines (showing on screen 
only one line for each set of duplicated lines). If the
does not exist, the program must say so. 

\section{Temperature averages}
\label{sec:temperature-averages}

Write a program that reads a file from disk in comma-separated
format~(CSV). Every line must contain a list of number separated by
commas. 

The program must output the average for every line plus the average
for the whole file. A file may look like the following: 

\begin{verbatim}
    25, 24, 20, 18, 15, 13, 14, 13, 15, 17, 19, 21 
    25, 25, 24, 20, 18, 15, 13, 14, 13, 15, 17, 19 
    21, 25, 25, 24, 20, 18, 15, 13, 14, 17, 19, 21 
    25, 25, 24, 20, 18, 15, 13, 14, 13, 15, 17, 19 
    21, 25, 25, 24, 20, 18, 15, 13, 14, 13, 15, 17 
    21, 25, 25, 24, 20, 18, 15, 13, 14, 13, 15, 17 
    19, 21, 25, 25, 24, 20, 18, 15, 13, 14, 13, 15 
    17, 19, 21, 25, 25, 24, 20, 18, 15, 13, 14, 13
    ...
\end{verbatim}

\section{Binary cp (**)}
\label{sec:binary-cp}

Write a program that takes two names from the user at the command
line. If a file with the first name exists, the program must copy it
into a file with the second name. If the first file does not exist,
the program must say so. If the second file does exists, the program
must ask the user whether to overwrite it or not, and act accordingly.

This is the same exercise as above with an important difference: it
must be able to copy binary files (use \verb+InputStream+ instead of
\verb+Reader+, etc). Try it with \verb+.class+ and \verb+.exe+ files
and check that the copies work exactly like the originals.

\end{document}