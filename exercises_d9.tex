\documentclass{article}
\usepackage[margin=2cm]{geometry}
\usepackage[dvips]{graphicx}
\begin{document}

\section*{Learning goals}
\label{sec:learning-goals}

Before the next day, you should have achieved the following learning
goals: 

\begin{itemize}
\item Understand the importance of automated testing. 
\item Write your own tests for classes already defined. 
\item Execute your own tests for classes already defined. 
\end{itemize}

You should be able to finish most of non-star exercises in the lab. 
Remember that star exercises are more difficult. 
\textbf{Do not try star-exercises unless the other ones are clear to
  you}.  


\section{Install JUnit 4}
\label{sec:install-junit-4}

Use your favourite search engine to find JUnit (``JUnit download''
will do, and probably just ``JUnit'' will do too). 

\begin{itemize}
\item Download the last stable version of JUnit 4.
\item Unzip it in your drive. You will find a JAR file, with a name
  similar to \verb+junit-4.10.1.jar+.
\item Place the JAR file in a place where you can find it. It is
  probably a good idea to create a folder ``lib'' where you place all
  the external libraries that you use (e.g.~\verb+h:\lib+ on windows
  or \verb+/opt/lib+ on Unix systems).
\item Now you can add the JAR file to your classpath at the command
  line (as in the notes) or by modifying the environment variable
  CLASSPATH. 
\end{itemize}

\section{Practice "Find bugs once"}
\label{sec:practice-find-bugs}

The method \verb+getInitials(String)+ has a bug! If you introduce a
name with more than space between words, it throws an exception.

Create a class that contains the method \verb+getInitials(String)+ as
described in the notes. Create also the test class as described in the
notes. 

Then follow the ``find bugs once'' algorithm: reproduce the bug manually,
reproduce the bug programmatically by adding a new test to the testing
class, then fix the bug and check that all tests pass. 



\end{document}