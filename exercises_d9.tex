\documentclass{article}
\usepackage[margin=2cm]{geometry}
\usepackage[dvips]{graphicx}
\begin{document}

\section*{Learning goals}
\label{sec:learning-goals}

Before the next day, you should have achieved the following learning
goals: 

\begin{itemize}
\item Understand the importance of automated testing. 
\item Write your own tests for classes already defined. 
\item Execute your own tests for classes already defined. 
\end{itemize}

You should be able to finish most of non-star exercises in the lab. 
Remember that star exercises are more difficult. 
\textbf{Do not try star-exercises unless the other ones are clear to
  you}.  


\section{Install JUnit 4}
\label{sec:install-junit-4}

Use your favourite search engine to find JUnit (``JUnit download''
will do, and probably just ``JUnit'' will do too). 

\begin{itemize}
\item Download the last stable version of JUnit 4.
\item Unzip it in your drive. You will find a JAR file, with a name
  similar to \verb+junit-4.10.1.jar+.
\item Place the JAR file in a place where you can find it. It is
  probably a good idea to create a folder ``lib'' where you place all
  the external libraries that you use (e.g.~\verb+h:\lib+ on windows
  or \verb+/opt/lib+ on Unix systems).
\item Now you can add the JAR file to your classpath at the command
  line (as in the notes) or by modifying the environment variable
  CLASSPATH. 
\end{itemize}

\section{Practice "Find bugs once"}
\label{sec:practice-find-bugs}

The method \verb+getInitials(String)+ has a bug! If you introduce a
name with more than space between words, it throws an exception.

Create a class that contains the method \verb+getInitials(String)+ as
described in the notes. Create also the test class as described in the
notes. 

Then follow the ``find bugs once'' algorithm: reproduce the bug manually,
reproduce the bug programmatically by adding a new test to the testing
class, then fix the bug and check that all tests pass. 


  % - test methods in hospital

  % - test methods in circular list

  % - test methods in doubly linked list

\section{Test implementations of a given interface}
\label{sec:test-impl-given}

You already know that an interface is a way of describing the
behaviour of the class without knowing the details. Sometimes, one
party provides the interface of a component and the other party
implements the interface. This is very common in big projects, where
small teams of programmers make parts of a bigger program (e.g.~web
browsers, word processors, multiplayer games), and the different
modules need to communicate with each other. 

The first party does not only define the interface, it also implements
the tests that the implementation (i.e.~the class that implements the
interface) must pass. (When this approach is taken to its logical
conclusion, we are talking about Test-Driven Development as we will
see very soon). 

Take the role of a project leader and implement the tests for two of
the interfaces you have implemented in past weeks. 

\subsection{Stack}
\label{sec:stack3}

The notes from Day~7 implemented a Stack interface in two different
ways. Create a battery of tests that verify that the classes
implementing the interface is working as expected. Use it to test both
implementations.

\subsection{Queue}
\label{sec:stack2}

You implemented a Queue interface ---maybe in two different ways--- on
Day~7. Create a battery of tests that verify that the class(es)
implementing the interface work/s as expected. 

\subsection{Set (*)}
\label{sec:stack5}

You implemented a Set interface ---possibly in two different ways--- on
Day~8. Create a battery of tests that verify that the class(es)
implementing the interface work as expected. 



\section{Testing mathematical functions}
\label{sec:test-math-funct}

On Day~7 you implemented a simple hash. Write a battery of tests to
verify its behaviour, paying special attention to border cases. 

Hint: Implement a loop that tries a fair amount of random numbers
(around two thousand, for the purposes of this exercise) and verify
that the output is within the range. 


\section{Testing dynamic structures}
\label{sec:testing-maps}

Write batteries of tests to verify the functionality of the dynamic
structures you have created in past weeks: 

\begin{itemize}
\item doubly-linked list (day 6)
\item circular list (day 6)
\item simple map (day 7)
\item sorted list (day 6)
\end{itemize}

Make sure that you test border cases, including situations like: 

\begin{itemize}
\item Adding the first element.
\item Removing the last element.
\item Adding the first element and then removing it\ldots and then
  adding another one.
\end{itemize}

  % - a list leave a pointer loose

  % - a list adds -1 so length() returns the wrong size

\section{More tests (*)}
\label{sec:more-tests-}

If you have finished with the other exercises, write additional
batteries of tests for other programs (in particular, the exercises
marked with a star)  that you have written in past
weeks. Some exercises that provide a harder challenge to test properly
are: 

\begin{itemize}
\item the anti-aircraft game from day 5
\item any of the sort algorithms from day 6
\item any of the unfair queues from day 7
\item the hash-table (day 7)
\item deletion of elements in a tree (day 8)
\item re-balancing of a tree (day 8)
\item the abstract syntax tree (day 8)
\item the pseudo-git tree (day 8) (**)
\end{itemize}

\end{document}