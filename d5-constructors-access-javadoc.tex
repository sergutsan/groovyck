
\section{More on clases}
\label{sec:more-clases}

There are two important aspects of classes that we have to learn
about. I am talking about constructor methods and levels of access. 

\subsection{Constructor methods}
\label{sec:constructor-methods}

When classes have been used in the preceding sections, they were used
in two steps: first the memory was reserved for them using \verb+new+
and then their fields were initialised. 

\begin{verbatim}
    Point corner = new Point();
    corner.x = 4;
    corner.y = 0;
\end{verbatim}

This is not too cumbersome unless you have a class that has much more
than two fields that need initialising. For example, if you wanted to
create a \verb+Person+ with first name, family name, gender, age, job,
nationality, etc, you would need a lot of code every time you created
a new \verb+Person+. 

\begin{verbatim}
    Person john = new Person();
    john.firstName = "John";
    john.familyName = "Smith";
    john.gender = Gender.MAN; // This is an enumerated type
    john.age = 30;
    // the rest of the parameters come here...
    Person mary = new Person();
    mary.firstName = "Mary";
    mary.familyName = "Jordan";
    mary.gender = Gender.WOMAN; // This is an enumerated type
    mary.age = 33;
    // the rest of the parameters come here...
\end{verbatim}

We have written a lot of code and we have just created two
objects. This is really boring. On top of that, it looks like we are
repeating code over and over again by having to initialise all fields
manually. There is a better way of doing this. 

Every class (or complex type) can have one or more \emph{constructor
  methods}. A constructor method is a special type of method that is
used to initialise an object of a class when it is first created, and
it is executed every time a new method is created using \verb+new+. In
other words, the constructor method is a way of telling Java to use
\verb+new+ to allocate the memory \emph{and} initialise the object
using only one line. 

A constructor method looks like a method without a return type (not
even \verb+void+). Have a look at this example:

\begin{verbatim}
class Point {
    double x;
    double y;
    
    Point(double x, double y) {
        this.x = x;
        this.y = y;
    }
    
    double moveTo(Point remote) {
        this.x = remote.x;
        this.y = remote.y;
    }
    
    // more methods here...
}
Point point = new Point(1,1);
println "The point is now at " + point.x + "," + point.y
Point remotePoint = new Point(10,20);
point.moveTo(remotePoint);
println "The point is now at " + point.x + "," + point.y
\end{verbatim}

No need to initialise the points after creating them. The constructor
method does it for both of them. 
If a class has more than one constructor method, Java chooses the
right one according to the positional parameters. For example, we
could create a class Rectangle that has one or two parameters:

\begin{verbatim}
class Rectangle {
    int length;
    int width;
    Rectangle(int length, int width) {
        this.length = length;
        this.width  = width;
    }
    // This method creates a square, all sides equal
    Rectangle(int length) {
        this.length = length;
        this.width  = length;
    }
\end{verbatim}

Every class has at
least one constructor method. If it is not explicit ---as it has been
the case with all classes in the previous sections---, Java 
automatically adds an empty
constructor: a constructor method with no parameters that does not
initialise any field. 





%%% Local Variables:
%%% mode: latex
%%% TeX-master: "main"
%%% End:
