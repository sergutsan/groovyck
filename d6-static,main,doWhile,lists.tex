% TODO: We need to explain JARS

\section{Static data}
\label{sec:static-data}

% TODO: mira a ver si se te ocurre un ejemplo mejor que una constante
%   o un contador de objectos. 
%       Con los métodos es más fácil: pure functions

% Math
% System

\section{The main method}
\label{sec:main-method}

As we leave Groovy behind and move into full-flavour Java, we need a
way to launch our program: a starting point. It is all too well to
have a lot of Java classes, each of them in their own \verb+.java+
file, but this is of little use if we cannot start our program from
the command like (or clicking on an icon, for that matter). 

In Groovy, we did not need an entry point. The programs or scripts
were their own entry point. We only needed to 
write \verb+groovy myScript.groovy+ and Groovy would
start the execution. The situation is slightly more complex in Java
because all files are classes, not scripts. We need to have an entry
point, a place in the code where Java knows the execution of the
program can start. This entry point is the \verb+main+ method. The
main method is a normal method of a class, and every class can have
one. It looks like this: 

\begin{verbatim}
    public static void main(String[] args) {
      // Here be the code to launch the program
    }
\end{verbatim}

If a class has a main method, it can be run from the operating system
by typing: 

\begin{verbatim}
    > java myClass
\end{verbatim}

assuming that there is a \verb+myClass.class+ file. The 
signature of the \verb+main+ method is a bit long, but we have
already learnt what each of its elements means: 

\begin{itemize}
\item It must be \verb+public+, because otherwise it cannot be
  accessed from outside the class (and how would the program launch
  then?).
\item It must be \verb+static+, because it pertains to the class, not
  to each of its objects.
\item It returns \verb+void+. This is a method that is never called by
  other classes, so a return type is unnecessary.
\item The only parameter is a 1-D array of Strings, called \verb+args+
  (short for arguments). Actually the name is not important. The
  elements in the array are the parameters passed in the command line,
  if any. For example, if you run a program with a line like
  \verb+java myClass www.google.com 80 true+, the first element
  (i.e.~args[0]) of \verb+args+ will be ``www.google.com'', the second
  element will be ``80'', and the third one will be ``true''. If you
  want to use these parameters with a different type you need to parse
  them. 
\end{itemize}

\subsection{Liberating your main method from static restrictions}
\label{sec:liber-your-launch}

As static code does not understand anything about instance
(i.e.~non-static) methods or data, you will not be able to use your
instance data from your main method. This seems restricting at first,
as in the following example: imagine that we want to start a program
from a BitTorrentDownloader class, and we want to give the name of the
host and the port from the command line

\begin{verbatim}
    public class BitTorrentDownloader {
        private String host;
        private int port;
    
        public BitTorrentDownloader(String host, int port) {
            this.host = host;
            this.port = port;
        }
    
        public static void main(String[] args) {
            host = args[0];
            port = Integer.parseInt(args[1]);
        }
    }
\end{verbatim}

When you try to compile this program, java will complain with the
following error: 

\begin{verbatim}
    non-static variables cannot be referenced from a static context
    host = args[0];
    ^
    port = Integer.parseInt(args[1]);
    ^
\end{verbatim}

One (bad) solution is to make \verb+host+ and \verb+port+ static, but
that means that there can only be one of each. This is usually a bad
idea in the long run. Maybe in a future version you will like several
BitTorrentDownloaders working in parallel, or maybe you would to
launch many at the same time and put them in a queue. All of this
would be impossible if the fields were static. Remember: \verb+static+
means ``only one per class'', and programmer usually like to have as
many as possible of everything, just in case. As a rule of thumb,
\textbf{never use \verb+static+ unless it is absolutely necessary}.

There is another (much better) solution: to instantiate the class
inside it main method, and then run all the code from a non-static
launching method. This method is commonly called \verb+run()+. See the
example: 

\begin{verbatim}
    public class BitTorrentDownloader {
        private String host;
        private int port;
    
        public BitTorrentDownloader(String host, int port) {
            this.host = host;
            this.port = port;
        }
    
        public static void main(String[] args) {
            String host = args[0];
            int port = Integer.parseInt(args[1]);
            BitTorrentDownloader downloader = new BitTorrentDownloader(host, port);
        }
    }
\end{verbatim}

%%% Local Variables:
%%% mode: latex
%%% TeX-master: "main"
%%% End:
