\section{Extra exercises}
\label{sec:additional-exercises}

This section provides some additional exercises for you, for more
practice with Groovy.  

\subsection*{Exercise X1}

What does this program do?

\VerbatimInput[frame=single,label=Example]{src/s3Example6.groovy}

\subsection*{Exercise X2}

If you have worked out what the above program does, can you
see that, for certain series of numbers, it will not produce the correct
output?  In what circumstances will it not work correctly, and how could
you change the program to make it work properly?

\subsection*{Exercise X3}

Write a program that takes a series of numbers (ending in 0) and counts
the number of times that the number 100 appears in the list.  For example,
for the series 2, 6, 100, 50, 101, 100, 88, 0, it would output 2.

\subsection*{Exercise X4}

Write a program that takes a series of lines of text and,
at the end, outputs the longest line.
You may assume there is at least one line in the input.

\subsection*{Exercise X5 (this one is a bit harder)}

Write a program that takes
a series of numbers.  If the
current number is the same as the previous number, it says ``Same'';
if the current number is greater than the previous one, it says ``Up'',
and if it's less than the previous one, it says ``Down''.
It makes no response at all to the
very first number.  For example, its output for the list 9, 9, 8, 5, 10, 10,
would be Same, Down, Down, Up, Same
(comparing, in turn, 9 and 9, 9 and 8, 8 and 5, 5 and 10, 10 and 10).
You may assume there are at least two numbers in the input.


%%% Local Variables:
%%% mode: latex
%%% TeX-master: "main"
%%% End:

