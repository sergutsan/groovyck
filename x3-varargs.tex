
\section{Variable arguments}
\label{sec:variable-arguments}

Sometimes a method requires an unknown number of parameters. For
example, you may have a method \verb+add(...)+ that took several
integers and returned the addition of them all. You already know two
different approaches for implementing such a method: the first one
involves the use of lists\ldots

\begin{verbatim}
    int add(List<Integer> operands) { ... }
\end{verbatim}

\ldots while the second one involves arrays\ldots

\begin{verbatim}
    int add(int[] operands) { ... }
\end{verbatim}

Both approaches are a bit clunky for programmers that want to use
them: they must create the list or array, populate it with data, and
then pass it as a parameter. 
%
Since Java 5 there is another way called \emph{varargs} (short for
\emph{var}iable \emph{arg}ument\emph{s}). Varargs is nothing more than
syntantic sugar\footnote{The term \emph{syntactic sugar} is used in
  programming to refer to those features of a programming language
  that make the programmer's life easier (both writing and reading)
  but do not represent a difference in design or paradigm with respect
  to other similar languages or to former versions of the same
  language. Boxing/unboxing, varargs, and the \emph{for--each} loop are 
  examples of syntactic sugar in Java.} for the array version
that allows the programmer to 
just add the operands on the method call. 

\subsection{How to use varargs}
\label{sec:how-use-varargs}

A method with a variable number of arguments is defined using
triple-dot notation: 

\begin{verbatim}
    public int add(int... operands) {
        ...
    }
\end{verbatim}

It is important to note that varargs can only exist at the end of the
list of parameters. 

\begin{verbatim}
    /** 
     * A generalisation of add(...) that admits any operation 
     * (as long as it is commutative) and any number of operands. 
     */
    public int apply(Operation op, Operand... numbers) {
        ...
    }

    /*
     * This is not valid Java. It will not compile. 
     */
    public int add(Operation... op, Operand... numbers) {
        ...
    }

    /*
     * This is not valid Java. It will not compile. 
     */
    public int add(Operand... numbers, Operation op) {
        ...
    }
\end{verbatim}

Calling a method with varargs is as easy as calling any other
method. The following are all valid invocations of
\verb+add(...)+. Note that the number of varargs may be zero, as in the
last example. 

\begin{verbatim}
    int result = add(1);
    int result = add(1, 2, 3, 4, 5);
    int result = add(1, 2, 3, 4, 5, 5, 4, 3, 4, 10);
    int result = add();
\end{verbatim}

From the point of view of the method, the varargs are just a normal
array: 

\begin{verbatim}
    public int add(int... operands) {
       int result = 0;
       for (int i = 0; i < operands.length; i++) {
           result += operands[i];
       } 
       return result;
    }

    // another possibility, using a for-each loop for more clarity
    public int add(int... operands) {
       int result = 0;
       for (int operand : operands) {
           result += operand;
       } 
       return result;
    }
\end{verbatim}

\subsection{When to use varargs in methods you create}
\label{sec:when-use-varargs}

Varargs are useful for any situation in which a method may accept an
arbitrary number of objects. It is very convenient in the case
of methods that accept arrays, especially in the case of generic
arrays. 

You should use varargs with care, always trying to make your code
clearer. For example, if your method requires two lists, it is
probably better to use two lists intead of a list and an
array/varargs. It is also a bad idea to overload a varargs method
because it may be difficult to figure out (from the caller's point of
view) which version of the method is being called. 