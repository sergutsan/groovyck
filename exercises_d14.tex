\documentclass{article}
\usepackage[margin=2cm]{geometry}
\usepackage[dvips]{graphicx}
\begin{document}

\section*{Learning goals}
\label{sec:learning-goals}

Before the next day, you should have achieved the following learning
goals: 

\begin{itemize}
\item Implement divide-and-conquer solutions to problems. 
\end{itemize}

\section{Binary search}
\label{sec:binary-search}

The most basic example of divide-and-conquer strategies is the binary
search. This is used to look for an element in a sorted list. 

We can find an element in a list by traversing through the whole list
and checking whether each element is the one we are looking for. The
number of comparisons that we need by using this algorithm is
proportional to the length of the list. 

If we know that the list is sorted, we can do better by divide and
conquer, by repeating these steps: 

\begin{description}
\item[: ] If the list is empty, it does not contain the
  element. If it is not empty, check the middle element, i.e.~the
  element at \verb+list.size()/2+. If it is the element we are looking
  for, we have finished. 
\item[Subproblem: ] If the element we are looking for is before the
  middle element, the next list to search into is the first half of
  the original list; otherwise, it is the second half.
\item[Integration: ] No need for integration in this case.  
\end{description}

Implement a binary search algorithm for a list of integer numbers. You
can use the classes in the Java Collection Library. 

% Exercises
%   - Greatest common divisor ( p > q => gcd(p,q) = gcd(q, p % q) )
%   - Koch star?
%   - Memoization: Fibonacci numbers
%   - Increments and unfolding
%   - Dominoes
%   - Finding longest common subsequence
%
%
%   - Finding a number
%   - Quicksort
%   - Merge sort
%   - Root finding for a polynomial (one between -1000 and 1000 if
%       sign(f(min)) != sign(f(max)) , as many as possible for **)


\end{document}