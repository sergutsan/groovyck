
\documentclass{article}
\usepackage[margin=2cm]{geometry}
\begin{document}

\section*{Learning goals}
\label{sec:learning-goals}

Before the next day, you should have achieved the following learning
goals: 

\begin{itemize}
\item Understand how to use interfaces in Java, and use them in your
  programs. 
\item Understand how stacks, queues, and maps work. 
\item Strengthen your understanding of pointers, and how they are used
  in dynamic data structures. 
\end{itemize}

You should be able to finish most of non-star exercises in the lab. 
Remember that star exercises are more difficult. 
\textbf{Do not try star-exercises unless the other ones are clear to
  you}.  


% TODO: add more trivial exercise to scaffold the supermarket

% TODO: reword the supermarket to make it clear that a supermarket is not a queue?

% TODO: provide additional code to make the Person, KidPerson, and AdultPerson compile and run

\section{Supermarket queue}
\label{sec:queues}

Use the interface \verb+PersonQueue+ that represents a queue of
people waiting at the supermarket and then implement it. You can use
the definition and the implementations of \verb+StringStack+ as a
guide. You can reuse any version of class \verb+Person+ from past
days. For example, depending on your implementation, you may want to
use a version of \verb+Person+ with or without internal pointers. 

\begin{verbatim}
    public interface PersonQueue {
        /**
         * Adds another person to the queue.
         */
        void insert(Person person);

        /**
         * Removes a person from the queue.
         */ 
         Person retrieve();
    }
\end{verbatim}

Then create a class \verb+Supermarket+ that has two methods:
\verb+addPerson(Person)+ and \verb+servePerson()+. These methods must
call the appropriate methods of \verb+PersonQueue+. 

\section{Supermarket queue revisited (*)}
\label{sec:superm-queue-revis}

Implement the interface \verb+PersonQueue+ in a different way. Then check that it
works exactly the same without changing either the interface or your
class \verb+Supermarket+. 

\section{Foreign people, different queues (*)}
\label{sec:fore-people-diff}

Get a queue implementation from one of your colleagues. Use it and
check that it works exactly the same without changing either the
interface or your class \verb+Supermarket+. 

If it does not work, why is this?

\section{Unfair queue (*)}
\label{sec:unfair-queue-}

\subsection{Simple}
\label{sec:simple}

Implement the interface queue in a way that the person at the end
(i.e.~the person that is retrieved next time the method
\verb+retrieve()+ is called) is always the oldest person. 

\subsection{Clustered}
\label{sec:clustered}

Implement the interface queue in a way that the next person retrieved
is always a person over 65, if there is any in the queue; if not, it
must be a person over 18, if there is any in the queue. Inside each
age category, the behaviour of the queue is typical FIFO: first in,
first out. 

These two queues are examples of \emph{priority queues}, and are not
strictly FIFO: old people will always come out of the queue before
younger people, even if the youngsters came to the queue
first. Priority queues are more difficult to implement, but they are
also important in computing. For example, your hard disk uses a
priority queue to decide where to move next: if the disk's head is at
position 555 and the queue of requests is

$$4, 99, 234, 500, 101, 43, 881, 77$$

your disk may decide to move to position
500 to reduce movement, time, and energy consumption.

\section{Maps}
\label{sec:using-maps}

\subsection{Hash function}
\label{sec:hash-function}

Create a class \verb+HashUtilities+ that implements a simple hash
function \verb+shortHash(int)+ that takes any integer and returns an
integer between 0 and 1000. You can use the modulo operator and the
static function \verb+Math.abs(int)+ for obtaining the absolute value
of an integer. 

Note that \verb+shortHash(int)+ is a pure function (it 
does not have any side effects),
so it should be \verb+static+. Then you can call this method using the
name of the class like \verb+HashUtilities.shortHash(int)+. 

Every object in Java has the method \verb+hashCode()+, that returns an
\verb+int+. Test your hash function by passing the hash codes of
different objects and verifying that it always returns a number
between 0 and 1000, as in the following example:

\begin{verbatim}
    println "Give me a string and I will calculate its hash code";
    String str = System.console().readLine(); 
    int hash = str.hashCode();
    int smallHash = HashUtilities.shortHash(hash);
    System.out.println("0 < " + smallHash + " < 1000");
\end{verbatim}

\subsection{Simple map}
\label{sec:simple-map-1}

Create a class that implements the following interface of a simple map
from integers to strings. 

\begin{verbatim}
    /**
     * Map from integer to Strings
     */
    public interface SimpleMap {
        /**
         * Puts a new String in the map. 
         * 
         * If the key is already in the map, nothing is done.
         */
        void put(int key, String name);
    
         /**
          * Returns the name associated with that key, 
          * or null if there is none.
          */
         String get(int key);
    
         /**
          * Removes a name from the map. Future calls to get(key) 
          * will return null for this key unless another 
          * name is added with the same key.
          */
         void remove(int key);
    
         /** 
          * Returns true if there are no workers in the map, 
          * false otherwise
          */
         boolean isEmpty();
    }
\end{verbatim}


\subsection{Hash table (*)}
\label{sec:simple-map-2}

Create a class that implements the following interface of a simple map
from integers to strings: it is a one-to-many mapping. 
A similar map is used in some countries to
classify the citizens into groups for tax purposes (so that each
department has a limited number of citizens to examine and process). 

\begin{verbatim}
    /**
     * Map from integer to Strings: one to many
     */
    public interface SimpleMap {
        /**
         * Puts a new String in the map. 
         */
        String put(int key, String name);
    
         /**
          * Returns all the names associated with that key, 
          * or null if there is none.
          */
         String[] get(int key);
    
         /**
          * Removes a name from the map.
          */
         void remove(int key, String name);
    
         /** 
          * Returns true if there are no workers in the map, 
          * false otherwise
          */
         boolean isEmpty();
    }
\end{verbatim}

Hint: You can implement it with arrays or with linked lists. You do
\textbf{not} know in advance how many strings you will receive for
every key.  


    
% \section{National Insurance}
% \label{sec:simple-map}

% Help your country by using the map and the hash function you have
% implemented in the following sections to make a program that links
% National Insurance numbers and names of workers. You must create a
% class that implements the following interface: 

% \begin{verbatim}
%     /**
%      * A simplified program for managing the National Insurance system
%      */
%     public interface NationalInsurance {
%         /**
%          * Puts a new worker in the system. 
%          * 
%          * If the name is already in the system, nothing is done. 
%          */
%         String put(String name, );

%          /**
%           * Returns the code used for this worker
%           */
%          String getWorkerCode(String name);
    
%          /**
%           * Removes a worker from the map. Future calls to get(id) 
%           * will return null for this key (i.e. id) unless another 
%           * worker is added with the same key.
%           */
%          void remove(String name);
%     }
% \end{verbatim}

% Test your implementation by adding workers to your map, removing
% workers, etc. 




\end{document}
