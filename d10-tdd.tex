
\section{Constants}
\label{sec:constants}

Programs use data, lots of data. Data is stored in variables of simple
and complex types, the latter being pointer that point to objects in
the memory. We have used a lot of variables up to now. 

Sometimes, we know that a piece of data is not going to change over
the course of the program. Maybe it is a physical constant like the
speed of light, maybe it is a mathematical constant like $\pi$, maybe
it is another element of your program that you know will never
change. 

Moreover, you want your program not to allow any part
of the code (possibly \emph{buggy}) to modify it. This is achieved by
using the keyword \verb+final+. This keyword tells Java that the value
of a variable must never change from the moment it is initialised
until the program ends. See an example: 

\begin{verbatim}
    public static final double PI = 3.14159265359; 
\end{verbatim}

% Cross-refs
You remember that we said that static \emph{fields} should be used
sparingly, and mostly (or only) for constants. The keyword
\verb+final+ is used to declare an identifier as constant, and
therefore both keywords are frequently found together when applied to
fields. 

You may also remember that we mentioned that constants fields can be
made public, as there is no risk of someone modifying their values and
having undesirable side effects. If you ever make a field public in a
class that has methods, make sure you make it static and (especially)
final too.



TDD

@Test
   expects
   timeout

% \item Ultra-brief introduction to annotations in Java
% \item Do not confuse annotations with JavaDoc!! @throws, etc
% \item Annotations to cover:
%     @Test
%       timeout = 15000
%       expected = IllegalStateException.class
%     @Before, @BeforeClass
%     @After, @AfterClass
%     @Override
%     @SuppressWarnings
%     @Deprecated


plus JavaDoc