\section{Constants}
\label{sec:constants}

Programs use data, lots of data. Data is stored in variables of simple
and complex types (the latter being pointers that point to objects in
the memory, in the heap). We have used a lot of variables since we
started our journey into programming. 

Sometimes, we know that a piece of data is not going to change. 
Maybe it is a physical constant like the
speed of light, maybe it is a mathematical constant like $\pi$, or
maybe it something else. Any such piece of data in your program, that
is known to not change, is called a \emph{constant}. 

You do not want your program to allow any section
of the code to modify your constants, even by mistake. This is achieved by
using the keyword \verb+final+. This keyword tells Java that the value
of a variable must never change from the moment it is initialised
until the program ends. See an example: 

\begin{verbatim}
    public static final double PI = 3.14159265359; 
\end{verbatim}

By convention, constants are written all in capital
letters. Multi-word constants use underscores, as in: 

\begin{verbatim}
    public static final int SPEED_OF_LIGHT = 299792458;
\end{verbatim}

% Cross-refs
You remember that we said that static \emph{fields} should be used
sparingly, and mostly (or only) for constants. The keyword
\verb+final+ is used to declare an identifier as constant, and
therefore both keywords (\verb+static+ and \verb+final+)
are frequently found together when applied to constants: final ensures
the value cannot be changed while static ensures there is only one
field (a ``box'') for the whole class rather than one duplication in
every object of the class.

You may also remember that we mentioned that constant fields can be
made public, as there is no risk of anyone modifying their values
producing some undesirable side effect. If you ever make a field public in a
class that has methods, make sure you make it static and (especially)
final.

\begin{center}
{\large \textbf{If a field is public, it must be static and final.}}  
\end{center}

The only exception to this rule is for classes without
methods. Remember that such a class can be used as a way of
interchanging several pieces of data bundled together (as a return
value, for example). 

% TODO: next year, make a bigger point of how final for a complex
% types means just a constant pointer, not a constant object. 

\section{Documenting your code with JavaDoc}
\label{sec:docum-your-code}

% TODO: write this better

We have already seen how the Java library is completely documented
online. All classes are described on a web page that explains
what they are and what they do (i.e.~their
methods, explaining the parameters), and this information is online on
the Java API documenation or JavaDoc. You can find the JavaDoc easily
by typing ``Java API'' on your favourite search engine. 

The good news is that it is easy to document your own classes in the
same professional way as the core Java library. In order to do so, we
just need to write the comments of our code in a special way and then
use the program \verb+javadoc+. 

\subsection{How to write comments}
\label{sec:writing-comments}

As we briefly saw when we introduced interfaces, JavaDoc comments
start by \verb+/**+, end by \verb+*/+, and can span several
lines. JavaDoc comments can also tag additional information regarding
parameters or return values, as shown in the following example: 

\begin{verbatim}
    /**
     * An StringStack is a dynamic structure that can contain any number
     * of strings. 
     * 
     * New elements (i.e. Strings) can be put on top of the stack or
     * removed from the top of the stack. Only the element at the top of
     * the stack can be removed at any given time; to access a specific 
     * element, all elements on top of it must be removed (popped) first. 
     * 
     * This interface allows the user to replace some elements of the list
     * even if they are not accessible. 
     */
    public interface StringStack {
        /**
         * Put an element at the top of the stack. 
         *
         * @param newString the new string to be put
         */
        void push(String newString);
        /**
         * Removes the element at the top of the stack and returns it. 
         *
         * @return the element at the top of the stack
         */
        String pop();
        /**
         * Replaces every occurrence of one string in the stack
         * for another string. 
         *
         * @param oldString the string to be replaced
         * @param newString the new string to replace it with
         */
        void push(String oldString, String newString);
    }
\end{verbatim}

There are three important things to notice in this example: 

\begin{itemize}
\item All comments start with \verb+/**+ and end with \verb+*/+. It is
  customary to make each line in the comment start with a star too.
\item Parameters are documented with the tag \verb+@param+, followed
  by the name of the parameter, followed by the description.
\item Return values are documented with the tag \verb+@return+,
  followed by the description of what is returned. 
\end{itemize}

Comments should explain the functionality of your classes and methods
in a short and clear way. Comments should not explain the internal
implementation details unless it really makes a difference to other
programmers using those classes or methods. In other words, comments
should be as clear as possible, and as short as possible, in that
order of importance\footnote{The Java documentation itself is a good
  inspiration of what good comments should look like. Use it as an
  inspiration as well as a source of information.}. 

\subsection{How to create the online documentation}
\label{sec:how-create-online}

Creating the webpages with the documentation of your classes and
methods is very simple. Assumming that you have all your classes
documented (i.e.~with proper comments) in a folder called \verb+src+
and want to put all the web pages in a folder called \verb+www+, you
just need to run the following command: 

\begin{verbatim}
    > javadoc path/to/src/*java -d path/to/www/
\end{verbatim}

This will generate all the web pages, HTML files, CSS style sheets,
links, etc, for you on the \verb+www+ folder. The folder will contain
a file called \verb+index.html+ as a starting point. 

