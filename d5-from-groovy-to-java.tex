
\section{Introducing Java}
\label{sec:introducing-java}

The time has come to start using full-flavoured Java. This section
will introduce the main features that are new with respect to what we
have seen so far. We will make a gradual transition to the new
language. 

\subsection{Everything is a class}
\label{sec:everything-class}

In Java all the code comes inside a class. There is no class-less
code as in Groovy. 

Usually every class is defined in its own file, e.g.~a \verb+Person+
class is defined in a file called \verb+Person.java+. When a file
contains only one class, this class must be public and have the same
name as the file (without the \verb+.java+ extension). 

Classes in different files must be compiled independently before they
can be used. This is another important difference with the Groovy
scripts we have been using until now, where every script was
self-contained. From now on we will have classes in different files,
and unless we compile\footnote{If you do not remember clearly what
  \emph{compiling} means, read again the notes from day 1. } them and
transform those text files into something that can be understood by
the machine, the machine will not find the classes you are
calling. Java classes (i.e.~files) are compiled with the java
compiler, \verb+javac+. 

\begin{verbatim}
    > javac Person.java
\end{verbatim}

This will produce a file \verb+Person.class+ in your folder. Once you
have it, you can use the Person class from any other file. 

\paragraph{Classpath. }
\label{sec:classpath.-}

You may be wondering how is it possible to use classes like String
that are not in your current directory in the form of a \verb+.class+
file. This is because Java comes with a lot of classes built-in,
including String. These classes are in your CLASSPATH, which is a list
of locations in your computer\footnote{A CLASSPATH is something
  conceptually similar to a PATH, a list of locations in your computer
  where the operating system looks for executable files when you type
  them in the command line. Both are environment variables, and can be
  accessed and modified in the same way.} where Java looks for the
classes you are using in your program. We will see more about this in
the future.

%%% Local Variables:
%%% mode: latex
%%% TeX-master: "main"
%%% End:
