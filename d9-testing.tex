
\section{Software testing}
\label{sec:software-testing}

I know your code has bugs. If it is a small project it will have few
bugs. If it is a big projects it has a lot of bugs. I know.

How do I know? Because I know you are a human being, and human beings
make mistakes. When a programmer is writing a program, there are too
many things to keep in the short-term memory at the same time: data
structures, communication between different parts of the program,
flows of information\ldots Any program except the most trivial one is
too complex for the human mind to see in its entirety. This is why we
need to use methods to isolate operations that are executed often and
why we use classes to group different functionalities of our program
around some conceptual ideas that we can grasp (e.g.~a supermarket has
queues, a queue has people, a person has a name and knows who is
behind in the queue; a company has products and employees, employees
have names and take money from the company, products give money to the
company). 

Structured and object-oriented programming are ways in which
programmers can reduce the complexity of their programs to levels that
are more or less manageable, but it is still impossible to write
programs that do not contain bugs. Humans, even the best programmers,
have a tendency to forget some details of the program. That is the way
the human mind works: it concentrates on the bug fundamental aspect
and forgets the details until needed\ldots which in computing means
when the program is already executing and nothing can be done about
it. 

Long story short, as long as programming is performed by humans,
programs will have bugs. The compiler can make sure that a program is
syntactically correct but it cannot tell if  it makes sense or
even whether it will do what the programmer expects. 

That is why software testing is important. Testing a program is a way
to ensure that the program does what it should. There are two types of
testing: manual and automatic (Table~\ref{tab:test}). You are already
familiar with the former type, you have been doing it for weeks with
the programs you have been writing until now. Now we will learn how to
do things properly in an automated way. 

\begin{table}[htbp]
  \centering
  \begin{tabular}{l|c|c}
    & Manual & Automatic \\
    \hline
    Thouroughness & Tests more bugs & Tests all bugs known \\
    Speed & Slow & Fast \\
    Experience & Very boring & The computer does the hard work \\
  \end{tabular}
  \caption{Manual testing vs automated testing}
  \label{tab:test}
\end{table}



You are human. Your code is buggy.

Table hand-testing vs automated-testing





Caso real: 

  - me dijiste que me dabas empty[] y me has dado un null!


  - es el mismo buig que hace un año! Me prometiste que no volvería a
  ocurrir!
    No es culpa mía! Alguien debe haber cambiado el código!



%%% Local Variables:
%%% mode: latex
%%% TeX-master: "main"
%%% End:
